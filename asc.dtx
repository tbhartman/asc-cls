% \iffalse meta-comment
% 
% Copyright (c) 2012 by Tim Hartman
% 
% This file may be distributed and/or modified under the
% conditions of the LaTeX Project Public License, either
% version 1.2 of this license or (at your option) any later
% version. The latest version of this license is in:
%
% http://www.latex-project.org/lppl.txt
%
% and version 1.2 or later is part of all distributions of
% LaTeX version 1999/12/01 or later.
%
% \fi
%
% \iffalse
%<*driver>
\ProvidesFile{asc.dtx}
%</driver>
%<class>\NeedsTeXFormat{LaTeX2e}[1999/12/01]
%<class>\ProvidesClass{asc}
%<*class>
   [2012/07/13 v0.1 ASC conference document class]
%</class>
%
%<*driver>
\documentclass{ltxdoc}
\EnableCrossrefs
\CodelineIndex
\RecordChanges
\usepackage[colorlinks,%
            linkcolor=black,%
            citecolor=black,%
            filecolor=black,%
            urlcolor=black]{hyperref}
\usepackage{amsmath}
\begin{document}
  \DocInput{asc.dtx}
\end{document}
%</driver>
% \fi
%
% \CheckSum{0}
%
% \CharacterTable
%  {Upper-case    \A\B\C\D\E\F\G\H\I\J\K\L\M\N\O\P\Q\R\S\T\U\V\W\X\Y\Z
%   Lower-case    \a\b\c\d\e\f\g\h\i\j\k\l\m\n\o\p\q\r\s\t\u\v\w\x\y\z
%   Digits        \0\1\2\3\4\5\6\7\8\9
%   Exclamation   \!     Double quote  \"     Hash (number) \#
%   Dollar        \$     Percent       \%     Ampersand     \&
%   Acute accent  \'     Left paren    \(     Right paren   \)
%   Asterisk      \*     Plus          \+     Comma         \,
%   Minus         \-     Point         \.     Solidus       \/
%   Colon         \:     Semicolon     \;     Less than     \<
%   Equals        \=     Greater than  \>     Question mark \?
%   Commercial at \@     Left bracket  \[     Backslash     \\
%   Right bracket \]     Circumflex    \^     Underscore    \_
%   Grave accent  \`     Left brace    \{     Vertical bar  \|
%   Right brace   \}     Tilde         \~}
%
%
% \changes{v0.0}{2012/07/11}{Initial version.}
%
% \DoNotIndex{\',\.,\@M,\@@input,\@Alph,\@alph,\@addtoreset,\@arabic}
% \DoNotIndex{\@badmath,\@centercr,\@cite}
% \DoNotIndex{\@dotsep,\@empty,\@float,\@gobble,\@gobbletwo,\@ignoretrue}
% \DoNotIndex{\@input,\@ixpt,\@m,\@minus,\@mkboth}
% \DoNotIndex{\@ne,\@nil,\@nomath,\@plus,\roman,\@set@topoint}
% \DoNotIndex{\@tempboxa,\@tempcnta,\@tempdima,\@tempdimb}
% \DoNotIndex{\@tempswafalse,\@tempswatrue,\@viipt,\@viiipt,\@vipt}
% \DoNotIndex{\@vpt,\@warning,\@xiipt,\@xipt,\@xivpt,\@xpt,\@xviipt}
% \DoNotIndex{\@xxpt,\@xxvpt,\\,\ ,\addpenalty,\addtolength,\addvspace}
% \DoNotIndex{\advance,\ast,\begin,\begingroup,\bfseries,\bgroup,\box}
% \DoNotIndex{\bullet}
% \DoNotIndex{\cdot,\cite,\CodelineIndex,\cr,\day,\DeclareOption}
% \DoNotIndex{\def,\DisableCrossrefs,\divide,\DocInput,\documentclass}
% \DoNotIndex{\DoNotIndex,\egroup,\ifdim,\else,\fi,\em,\endtrivlist}
% \DoNotIndex{\EnableCrossrefs,\end,\end@dblfloat,\end@float,\endgroup}
% \DoNotIndex{\endlist,\everycr,\everypar,\ExecuteOptions,\expandafter}
% \DoNotIndex{\fbox}
% \DoNotIndex{\filedate,\filename,\fileversion,\fontsize,\framebox,\gdef}
% \DoNotIndex{\global,\halign,\hangindent,\hbox,\hfil,\hfill,\hrule}
% \DoNotIndex{\hsize,\hskip,\hspace,\hss,\if@tempswa,\ifcase,\or,\fi,\fi}
% \DoNotIndex{\ifhmode,\ifvmode,\ifnum,\iftrue,\ifx,\fi,\fi,\fi,\fi,\fi}
% \DoNotIndex{\input}
% \DoNotIndex{\jobname,\kern,\leavevmode,\let,\leftmark}
% \DoNotIndex{\list,\llap,\long,\m@ne,\m@th,\mark,\markboth,\markright}
% \DoNotIndex{\month,\newcommand,\newcounter,\newenvironment}
% \DoNotIndex{\NeedsTeXFormat,\newdimen}
% \DoNotIndex{\newlength,\newpage,\nobreak,\noindent,\null,\number}
% \DoNotIndex{\numberline,\OldMakeindex,\OnlyDescription,\p@}
% \DoNotIndex{\pagestyle,\par,\paragraph,\paragraphmark,\parfillskip}
% \DoNotIndex{\penalty,\PrintChanges,\PrintIndex,\ProcessOptions}
% \DoNotIndex{\protect,\ProvidesClass,\raggedbottom,\raggedright}
% \DoNotIndex{\refstepcounter,\relax,\renewcommand}
% \DoNotIndex{\rightmargin,\rightmark,\rightskip,\rlap,\rmfamily}
% \DoNotIndex{\secdef,\selectfont,\setbox,\setcounter,\setlength}
% \DoNotIndex{\settowidth,\sfcode,\skip,\sloppy,\slshape,\space}
% \DoNotIndex{\symbol,\the,\trivlist,\typeout,\tw@,\undefined,\uppercase}
% \DoNotIndex{\usecounter,\usefont,\usepackage,\vfil,\vfill,\viiipt}
% \DoNotIndex{\viipt,\vipt,\vskip,\vspace}
% \DoNotIndex{\wd,\xiipt,\year,\z@}
%
% \GetFileInfo{asc.dtx}
% \title{\LaTeX{} Class for Conference Papers submitted to the
%    American Society of Composites.\thanks{This
%    file has version number \fileversion, last revised \filedate.}}
%
% \author{Tim Hartman \\ \texttt{tbhartman@vt.edu}}
%
% \maketitle
%
% \begin{abstract}
%   This document provides a \LaTeXe{} class for formatting \LaTeX{} documents to be submitted to conferences hosted by the American Society of Composites.
%   This class is neither provided nor endorsed by the American Society of Composites.
%   Proceed with caution; this was written for the 2012 ASC Conference in Arlington, TX.
% \end{abstract}
%
% \tableofcontents
%
% \section{Introduction}
% 
% The American Society of Composites provides instructions for formatting papers submitted to its conferences.
% However, it provides no means for formatting \LaTeX{} documents.
% This \texttt{asc-cls} seeks to fill the gap by styling the \LaTeX{} document as well as replicating the  instructions.
% The instructions to which this document adheres may be
% found at
% \url{http://asc2012.uta.edu/ASC\%20AUTHOR\%20GUIDELINES\%202012.pdf}.
% 
% If you are working in Microsoft Word, please find a template at
% \url{http://asc2012.uta.edu/ASC\_AUTHOR\_GUIDELINES\_2012.doc}.
% 
%
% \section{Usage}
% 
% Two files are needed to format your \LaTeX\ document according to the ASC standards; the class file \texttt{asc.cls} and the bibliography style file \texttt{asc.bst}.
% To obtain these files, you must first have \texttt{asc.dtx}, \texttt{asc.ins}, and \texttt{asc.dbj}.
% To extract the \texttt{asc.cls} file, run \\[5pt]
% |   latex asc.ins|\\[5pt]
% To obtain this documentation, run \\[5pt]
% |   pdflatex asc.dtx|\\[5pt]
% To obtain \texttt{asc.bst}, run \\[5pt]
% |   latex asc.dbj|\\[5pt]
%
% The ASC class is invoked by including \\[5pt]
% |   \documentclass[|{\itshape options}|]{asc}|\\[5pt]
% at the beginning of your document.
% The bibliography style file \texttt{asc.bst} is needed for formatting the bibliography.
% Besure that is in your \TeX path and have\\[5pt]
% |   \bibliographystyle{asc}|\\[5pt]
% before the |\bibliography| command.

%
% \section{Options}
% 
% With no options given to the |\documentclass{asc}| call, the rendered document should adhere to ASC formatting guidelines.
% However, this formatting may not be most condusive to proofreading the manuscript.
% The following options alter the formatting of the |asc| class (and, thus, deviate from the formatting requirements).
% Be sure to remove all options before the final rendering.
%
% \subsection{Double spacing}
% \DescribeMacro{doublespacing}
% Add the |doublespacing| option to produce a double-spaced document.
% This multiplies the |\baselineskip| by $1.667$.
%
% \subsection{Margin notes}
% \DescribeMacro{marginnotessize}
% Adding |marginnotessize| to the options list renders the document with wide right margins.
% Additionally, the |\note| macro prints its optional argument to the margin (without |marginnotesize|, the optional argument to |\note| is not rendered).
%
% \subsection{Page numbers}
% \DescribeMacro{pageno}
% \DescribeMacro{nopageno}
% Adding |pageno| to the options list prints page numbers at the bottom of the each page (using the \LaTeX\ defaults).
% Conversely, |nopageno|, the default option, removes all page numbers per the formatting guidelines.
%
%
% \subsection{Title page}
%
% \DescribeMacro{\affiliation}
% \DescribeMacro{titlepage}
% \DescribeMacro{notitlepage}
% \DescribeMacro{toc}
% The default option |titlepage| changes the definition of |\maketitle| to match ASC standards.
% With |notitlepage|, |\maketitle| creates a default title page (contradicting the formatting required by ASC).
% This is currently buggy, so just let it to |titlepage|.
% The |toc| option works a table of contents in to the ASC formatted title page.
% This is non-standard.
% This option (|toc|) as no effect when |notitlepage| is given.
% I define the |\affiliation| macro which you use like |\author| or |\date|.
% It works, kind of.
%
% \StopEventually{\PrintChanges \PrintIndex}
%
% \section{Code Documentation}
%
% This section is dedicated to explaining the \texttt{asc.cls} code itself.
% 
% \subsection{Some things to start with}
%
% Several options are held in boolean variables, defined here:
%    \begin{macro}{\if@doublespacing}
%    \begin{macro}{\if@marginnotessize}
%    \begin{macro}{\if@pageno}
%    \begin{macro}{\if@asctitlepage}
%    \begin{macro}{\if@asctitlepagetoc}
%    A switch to indicate if a titlepage has to be produced.  For the
%    article document class the default is not to make a separate
%    titlepage.
%    \begin{macrocode}
\newif\if@doublespacing
\newif\if@marginnotessize
\newif\if@pageno
\newif\if@asctitlepage
\newif\if@asctitlepagetoc
%    \end{macrocode}
%    \end{macro}
%    \end{macro}
%    \end{macro}
%    \end{macro}
%    \end{macro}
%
%
% \subsection{Options}
% \subsubsection{Declaration of Options}
% Declare all the options available.
% Pass unknown options to article.
%    \begin{macrocode}
\DeclareOption*{\PassOptionsToClass{\CurrentOption}{article}}
\DeclareOption{doublespacing}{\@doublespacingtrue}
\DeclareOption{marginnotessize}{\@marginnotessizetrue}
\DeclareOption{pageno}{\@pagenotrue}
\DeclareOption{nopageno}{\@pagenofalse}
\DeclareOption{titlepage}{\@asctitlepagetrue}
\DeclareOption{notitlepage}{\@asctitlepagefalse}
\@asctitlepagetocfalse
\DeclareOption{toc}{\@asctitlepagetoctrue}
%    \end{macrocode}
%
% \subsubsection{Executing Options}
%
% We know process all the options, as well as defining the defaults.
%    \begin{macrocode}
\ExecuteOptions{nopageno,titlepage}
\ProcessOptions
%    \end{macrocode}
%
% \subsection{Load \texttt{Article} Class}
% 
% The article class is loaded with required options.
% I pass the letterpaper, but I could check later which is used and set the margins accordingly.
%    \begin{macrocode}
\LoadClass[12pt,letterpaper]{article}%
%    \end{macrocode}
%
% \subsection{Do the formatting!}
%    \begin{macrocode}
\RequirePackage{etoolbox}
\if@pageno\else
\RequirePackage{nopageno}
\fi
%    \end{macrocode}
%
%
% \subsubsection{Page Layout}
%    I could do this based on the size of the paper, but I'll just assume we have \texttt{letterpaper} and continue.
%    I account for the \texttt{marginnotessize} and \texttt{doublespacing} options here.
%    \begin{macrocode}
\if@marginnotessize
\RequirePackage[left=0.65in,right=2.23in,top=0.75in,bottom=0.93in,marginparwidth=1.8in]{geometry}
\else
\RequirePackage[left=1.32in,right=1.56in,top=0.75in,bottom=0.93in]{geometry}
\fi
\if@doublespacing
    \linespread{1.667}
\fi
%    \end{macrocode}
%
%    Use Times New Roman font.
%    \begin{macrocode}
\RequirePackage{times}
%    \end{macrocode}
%
%    \DescribeMacro{\note}
%    I define |\note| for making personal notes within a document, intended to be removed before the final rendering.
%    They must be removed manually at this point.
%    \begin{macrocode}
\RequirePackage{color}
\makeatletter
\def\note{\@ifnextchar[{\@with}{\@without}}
\def\@with[#1]#2{\PackageWarning{ASC}{note}\marginpar{\tiny \color{blue}#1} {\color{blue} #2}}
\def\@without#1{\note[]{#1}}
\makeatother
%    \end{macrocode}
%
%    \DescribeEnv{abstract}
%    A limited |abstract| environment is available.
%    I want it to be in the Table of Contents, but it should not be numberd.
%    \begin{macrocode}
\RequirePackage{calc}
\newlength{\ascabsskip}
\setlength{\ascabsskip}{2.75in-0.6in}
\renewenvironment{abstract}{\vspace*{\ascabsskip}\section*{Abstract}\addcontentsline{toc}{section}{Abstract}}{}
%    \end{macrocode}
%
%    Floats must be only at the top or bottom of the page, so set the defaults here.
%    \begin{macrocode}
\RequirePackage{float}
\floatplacement{figure}{tb}%
\floatplacement{table}{tb}%
%    \end{macrocode}
%    Furthermore, I change some default \LaTeX\ spacing options.
%    I don't really know what I'm doing, but it looks okay\ldots
%    \begin{macrocode}
\renewcommand{\topfraction}{0.9}
\renewcommand{\bottomfraction}{0.8}
\renewcommand{\textfraction}{0.07}
\renewcommand{\floatpagefraction}{0.7}
\renewcommand{\dblfloatpagefraction}{0.7}
%    \end{macrocode}
%
%    Floats must have space between them and the main text.
%    This seems to do the trick.
%    \begin{macrocode}
\setlength{\textfloatsep}{2.000\baselineskip}
%    \end{macrocode}
%
%    Paragraphs must be indented by $18 \text{~pt}$.
%    \begin{macrocode}
\AtBeginDocument{
  \setlength{\parindent}{18pt}%
}
%    \end{macrocode}
%
%    Spacing is prescribed between various sections of the document.
%    The \texttt{titlesec} package seems to be wonderful for this.
%    The options are declared by \\[5pt]
%    |\titlespacing| \marg{command} \marg{left sep} \marg{before sep} \marg{after sep} \marg{right sep} \\[5pt]
%    for spacing and \\[5pt]
%    |\titleformat| \marg{command} \oarg{shape} \marg{format} \marg{label} \marg{sep} \marg{before} \oarg{after}\\[5pt]
%    for the formatting.
%    \begin{macrocode}
\RequirePackage{titlesec}
\titlespacing
  {name=\section} % command
  {0pt} % left
  {2\baselineskip} % before sep
  {\baselineskip} % after sep
  %{} % right sep
\titleformat
  {name=\section} % command
  [hang] % shape
  {\bfseries} % format
  {\thesection} % label
  {1em} %sep
  {\uppercase} %before
  [] %after

\titlespacing
  {name=\subsection} % command
  {0pt} % left
  {\baselineskip} % before sep
  {\baselineskip} % after sep
  %{} % right sep
\titleformat
  {name=\subsection} % command
  [hang] % shape
  {\bfseries} % format
  {\thesubsection} % label
  {1em} %sep
  {} %before
  [] %after

\titlespacing
  {name=\subsubsection} % command
  {0pt} % left
  {\baselineskip} % before sep
  {\baselineskip} % after sep
  %{} % right sep
\titleformat
  {name=\subsubsection} % command
  [hang] % shape
  {} % format
  {\thesubsubsection} % label
  {1em} %sep
  {\uppercase} %before
  [] %after
%    \end{macrocode}
%
%    The bibliography section has some tweaks that I'm not sure how to deal with other than redefining it entirely.
%    This is copied from the \texttt{article} class definition.
%    The key parts are making the name be all caps and the references be $10 \text{~pt}$ font.
%    \begin{macrocode}
\renewenvironment{thebibliography}[1]
     {\section*{\refname}%
      \addcontentsline{toc}{section}{\refname}%
      \small
      \@mkboth{\MakeUppercase\refname}{\MakeUppercase\refname}%
      \list{\@biblabel{\@arabic\c@enumiv}}%
           {\settowidth\labelwidth{\@biblabel{#1}}%
            \leftmargin\labelwidth
            \advance\leftmargin\labelsep
            \@openbib@code
            \usecounter{enumiv}%
            \let\p@enumiv\@empty
            \renewcommand\theenumiv{\@arabic\c@enumiv}}%
      \sloppy
      \clubpenalty4000
      \@clubpenalty \clubpenalty
      \widowpenalty4000%
      \sfcode`\.\@m}
     {\def\@noitemerr
       {\@latex@warning{Empty `thebibliography' environment}}%
      \endlist}
%
%    Here I rework the caption formats for figures and tables using the \texttt{caption} package.
%    Both need a smaller font.
%    Tables need all caps, which is not a built-in option to \texttt{caption}, so I define a custom format with |\DeclareCaptionFormat|.
%    \begin{macrocode}
\RequirePackage{caption}
\captionsetup[figure]{font={small}}
\DeclareCaptionFormat{caps}{{#1}{#2}\uppercase{#3}\par}
\captionsetup[table]{font={small},format={caps},name=TABLE}
%    \end{macrocode}
%    Furthermore, tables should be referenced by roman numerals.
%    \begin{macrocode}
\renewcommand*\thetable{\Roman{table}}
%    \end{macrocode}
%
%    Here I allow the definition for an affiliation.
%    \begin{macro}{\affiliation}
%    \begin{macrocode}
%    \end{macrocode}
%    \end{macro}
%    Now we can make the title page.
%    \begin{macrocode}
\if@asctitlepagetoc
  \def\@makeasctitletoc{\tableofcontents}
\else
  \def\@makeasctitletoc{}
\fi
\if@asctitlepage
  \renewcommand\maketitle{\begin{titlepage}%
  \let\footnotesize\small
  \let\footnoterule\relax
  \let \footnote \thanks
  \null\vfil
  \vskip 60\p@
  \begin{center}%
    {\LARGE \@title \par}%
    \vskip 3em%
    {\large
     \lineskip .75em%
      \begin{tabular}[t]{c}%
        \@author
      \end{tabular}\par}%
      \vskip 1.5em%
    {\large \@date \par}%       % Set date in \large size.
  \end{center}
  \@makeasctitletoc\vfill\par
  \@thanks
  \vfil\null
  \end{titlepage}%
  \setcounter{footnote}{0}%
  \global\let\thanks\relax
  \global\let\maketitle\relax
  \global\let\@thanks\@empty
  \global\let\@author\@empty
  \global\let\@date\@empty
  \global\let\@title\@empty
  \global\let\title\relax
  \global\let\author\relax
  \global\let\date\relax
  \global\let\and\relax
}
\else
  \renewcommand\maketitle{%
  \global\let\thanks\relax%
  \global\let\maketitle\relax%
  \global\let\@maketitle\relax%
  \global\let\@thanks\@empty%
  \global\let\@author\@empty%
  \global\let\@date\@empty%
  \global\let\@title\@empty%
  \global\let\title\relax%
  \global\let\author\relax%
  \global\let\date\relax%
  \global\let\and\relax%
}
\fi
\makeatother
%    \end{macrocode}
%
%    That's all!
% \Finale
%
\endinput

